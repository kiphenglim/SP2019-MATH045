\documentclass[11pt,letterpaper,boxed]{hmcpset}

% include this package to customize page layout of document (e.g. to adjust margins)
\usepackage[margin=0.9in]{geometry}
\usepackage{ulem}

% set numbering style for enumerated lists to be of form (a), (b), (c), etc.
\renewcommand{\labelenumi}{{\bf (\alph{enumi})}}

%% Extra packages
\usepackage{amsfonts, amsmath, amssymb, enumerate, esvect, fancyhdr, gensymb, graphicx, lastpage, mathtools, mathrsfs, parskip}
%%

% Bold font shortcut
\renewcommand{\v}[1]{\textbf{#1}}

% info for header block in upper right hand corner
\name{Name: \underline{\hspace{4cm}}}
\class{Section: \underline{\hspace{1cm}}}
\assignment{Mailbox \#\underline{\hspace{2cm}}}
\duedate{April 5, 2019}

\begin{document}

\problemlist{Math 045: Homework 3}

%------------------------- Problem 1 -----------------------

\begin{problem}[1]
(5 points) A certain drug is being administered intravenously to a hospital patient who has had no prior drug treatments. Fluid containing 5 mg/cm$^3$ of the drug enters the patient’s bloodstream at a rate of 100 cm$^3$/hr. The drug is absorbed by tissues or otherwise leaves the bloodstream at a rate proportional to the amount present, with a rate constant of 0.4 (hr)$^{-1}$.
\begin{enumerate}
\item Assuming that the drug is always uniformly distributed throughout the bloodstream, write a differential equation for the amount of the drug that is present in the blood- stream at any time and state the initial condition.
\item Use the integrating factor method to solve this IVP.
\item How much of the drug is present in the bloodstream after a long time?
\end{enumerate}
\end{problem}

%\begin{solution}
%\end{solution}

\pagebreak

%------------------------- Problem 2 -----------------------

\begin{problem}[2]
(10 points total: 5 points for part (a) and 5 points for parts (bcd)) Consider the DE $y' = \cos(t)-y$.
\begin{enumerate}
\item Find the general solution to this ODE (by hand).
\item Generate the direction field and some solution curves for this DE with a range of different initial conditions. Google “slope field” or “direction field” to find an app that draws slope fields. Attach a printout with your homework.
\item Based on your numerical exploration in part (b), conjecture what happens to y(t) as t tends to infinity. Can you explain this behavior from the general solution you found in part (a)?
\item If you were given the sketch of the direction field from part (a), but were not given the equation, how would you know that the DE was non-autonomous?
\end{enumerate}
\end{problem}

%\begin{solution}
%\end{solution}

\pagebreak

%------------------------- Problem 3 -----------------------

\begin{problem}[3]
(10 points total: 5 points for parts (abc) and 5 points for (d)) You have been hired by a fishery to do some preliminary work on modeling a fish population under various harvesting strategies. They have asked you to start with this mathematical model that accounts for logistic growth, death, and harvesting:

\[\frac{dP}{dt} =rP(1-P/K)-\alpha P-H(t,P)\]

Here, $P(t)$ represents the fishery biomass (total mass of fish). The constant $r$ relates to the growth rate of your population, and $K$ is the carrying capacity of your fishery -- the maximum biomass that the fishery can sustain. The term $-\alpha P$ accounts for the natural death of the fish. Assume $r, K, \alpha,$ and $P(t)$ are all positive.

The function $H(t,P)$ describes the harvesting of the fish. The fishery wants to understand what would happen to the population of fish under different harvesting scenarios.

\begin{enumerate}
\item Pick your favorite aquatic organism to model, then come up with reasonable values for all of your parameters. What units will they have? You can try to look up values on the Internet, or just make some reasonable estimates.
\item What initial condition(s) will you use? Explore how your solutions might differ with different choices of initial condition.
\item First, consider harvesting strategies that are time invariant. In other words, consider only cases where $H$ is not a function of time. Here are two natural choices for
$H = H(P)$:\\
• the fishery could harvest at a constant rate (say $H = 10$ kg/day)\\
• the fishery could harvest at a rate that is proportional to the fishery biomass (e.g. 1\% of the biomass is harvested every month).\\
In each of these two cases, calculate the long-term population of the fish, and the long-term harvesting rate. Perform these calculations by hand.
\item In the case where $H$ is proportional to $P$, compare the following:\\
• Solve for $P(t)$ analytically (by hand).\\
• Numerically approximate $P(t)$ using Euler’s Method, twice with different time steps. You may use a spreadsheet, Python, MATLAB, or other computer methods to perform Euler’s method. (Use MATLAB’s ode45 numerical DE solver to approximate $P(t)$. We are providing code for you to run in MATLAB (see the “fishcode” folder on Sakai)).\\
Compare the accuracy of each set of results relative to the analytic answer.
\item (Optional) Think of some other harvesting strategy that might be reasonable besides the two considered here. Your H (t, P ) could involve time. (Perhaps it changes with the season?) Extend your numerical method so that it can also numerically approximate P(t) for your chosen harvesting function.
\end{enumerate}
\end{problem}

%\begin{solution}
%\end{solution}

\pagebreak
\hfill
\pagebreak

%------------------------- Problem 4 -----------------------

\begin{problem}[4]
(5 points) Examine Student Y’s work on the following problem. What did the student do correctly? What mistake(s) did the student make? What is a more correct response to the problem?\\
Come up with as many different explanations to help Student Y as you can. At the least, one explanation should involve the fact that the DE is autonomous and another should involve slope fields.
\end{problem}
\begin{figure}[ht!]
\centering
\includegraphics[width=100mm]{p3.jpg}
\end{figure}

%\begin{solution}
%\end{solution}

\pagebreak

%------------------------- Problem 5 -----------------------

\begin{problem}[5]
\v{Leaky bucket}: (10 points; 5 points for (ab) and 5 points for (cd))A bucket in the shape of a cylinder has a small hole in its bottom, through which water is leaking. Let $h(t)$ be the height of the water in the bucket, A the cross-sectional area of the bucket, and a the area of the small hole. We want to find how long it will take the bucket to empty.
\begin{enumerate}
\item Let’s assume that the water flows out of the bucket laminarly, so that the stream of water looks roughly like a cylinder with cross-sectional area $a$ (the same as the hole).\\
Let $v(t)$ be the velocity of the water when it exits the bucket. It is a fact that
\[av(t) = A h'(t).\]
What physical principle is responsible for this fact?\\
\v{Hint}: In a small unit of time $\Delta t$, the height of the water in the bucket falls by $\Delta h$ and the water at the hole travels a distance $v(t)\Delta t$, as indicated in the diagram above.
\item Now let’s derive a differential equation for $h(t)$ based on conservation of energy. If the
height of the water in the bucket decreases by $\Delta h$ and the density of the water is $\rho $,
the net change of potential energy in the system is $\Delta mgh = \rho A\Delta hgh$. Since energy
cannot be created or destroyed, this decreased potential energy must be converted into
kinetic energy. The kinetic energy of the same amount of water leaving the bucket
is $\frac 12 \Delta m v^2 = \frac 12 \rho A \Delta h v^2$. Equating these two energies, we get $v^2 = 2gh$. Using this
information, derive a differential equation for $h(t)$.
\item Assume that at $t = 0$ the height of the water in the bucket is $h_0$. Define $t*$ to be the time when the bucket becomes empty. Find an expression for $t*$. What is $h(2t*)$? (Make sure your answer is sensible.)
\item Suppose that at $t = 5$ you observe that the bucket is empty. Is it possible to determine uniquely how full the bucket was at $t = 0$? Explain why or why not.
\end{enumerate}
\end{problem}
\begin{figure}[ht!]
\centering
\includegraphics[width=75mm]{p5.jpg}
\end{figure}

%\begin{solution}
%\end{solution}

\pagebreak
\hfill
\pagebreak

%------------------------- Problem 6 -----------------------

\begin{problem}[6]
(5 points) For each IVP, what can you say about the interval forward in time from the initial condition in which you can guarantee existence and uniqueness of the solution? Make sure you explain why your claim is true. Use dfield1 to plot a direction field and solution curve for each IVP. In a couple of sentences, explain how the solutions you see are or are not consistent with the results of the theorem(s) you applied. \v{You do not have to analytically solve the IVPs.}
\begin{enumerate}
\item $(\sin t)y' +y=2; y(5)=-2$
\item$(t - 5)y' + ty = e-t; y(0) = 1$
\item$ty' =y+ty; y(0)=0$
\end{enumerate}
\end{problem}

%\begin{solution}
%\end{solution}

\pagebreak

%------------------------- Problem # -----------------------

\begin{problem}[7]
(5 points) In this exercise, we will prove the famous angle sum identity for sine.
\begin{enumerate}
\item State the angle sum identity for sine. This is the one that gives an expression for $\sin(a + b)$. You may look it up if you do not remember it.
\item In order to write an ODE, we have to express each side of the identity as a function of a single independent variable. Treat $b$ as a fixed constant, and write each side of the equation as a function of $t$, i.e. $\sin(a + b)$ becomes $y_1(t) = \sin(t + b)$.
\item Find an IVP that is satisfied by $y_1(t)$.
\item Prove the angle sum identity. Hint: Consider $y_2 = \sin t \cos b + \cos t \sin b$. Does it satisfy
your IVP from part (c)? Use existence and uniqueness.
\end{enumerate}
\end{problem}

%\begin{solution}
%\end{solution}

\pagebreak

%------------------------- Problem 8 -----------------------

\begin{problem}[8]
(5 points) Determine which method(s), if any, can be used to solve the following ODEs. You
do not need to solve the ODE, but provide a brief explanation.
\begin{enumerate}
\item $y' + (\cos t)y = \cos t$
\item $\frac{dy}{dx} +xy^2 =e^x$
\item $y'' + y' \sin t + 3y^2 = t$
\item $e^xy' + y = \tan x$
\end{enumerate}
\end{problem}

%\begin{solution}
%\end{solution}

\end{document}