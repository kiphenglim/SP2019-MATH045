\documentclass[8pt,letterpaper,boxed]{hmcpset}
\usepackage[left=0.75cm, right=0.75cm, top=0.75cm, bottom=0.75cm, headheight=0pt,headsep=0pt]{geometry}
\usepackage{amsfonts, amsmath, amssymb, enumerate, fancyhdr, gensymb, lastpage, mathtools, parskip}

\pagestyle{empty}
\linespread{1}
\parindent=0in
\parskip=0pt

\begin{document}
\problemlist{Math 045 Differential Equations Reference Sheet}

%------------------------- Classifying ODE's -----------------------

\textbf{Classifying ODE's}
\begin{itemize}
    \item Order of an ODE: Highest number of derivatives of dependent variable.
    \item Autonomous/Non-Autonomous: Independent variable not explicitly in the equation.
    \item Linear/Non-Linear: Dependent variable and derivatives appear linearly in equation. No products of variables and derivatives occur.
    \item Driven/Undriven (Only if DE is Linear): Driven (homogeneous, forced): Nonzero term that depends on independent variable or is constant. $f(x) = 0$ is always a solution to an undriven ODE but never to a driven ODE.
\end{itemize}

%------------------------- Existence and Uniqueness -----------------------
\textbf{Existence and Uniqueness}
\begin{enumerate}
    \item 1st Order Linear ODEs
    \begin{itemize}
        \item Given ODE in normal form $y'(t) + p(t)y(t) = q(t)$ and IC $y(t_0 = y_0$: If $p(t)$ and $q(t)$ are continuous functions of $t$ on an interval $(a, b)$ that contains $t_0$, then a solution to IVP exists and is unique for all $t \in (a, b)$.
    \end{itemize}
    \item 1st Order ODEs
    \begin{itemize}
        \item Given ODE $y' = f(t, y)$ and IC $y(t_0) = y_0$: If $f(t, y)$ and $\frac{\delta f}{\delta y} (t, y)$ are continuous in a rectangle around $(t_0, y_0)$, then a solution to IVP exists and is unique in some interval $(t_0-h, t_0+h)$ for some unknown $h>0$.
    \end{itemize}
    \item 2nd Order Linear ODEs
    \begin{itemize}
        \item Given ODE in normal form $y''(t) + p_1(t)y'(t) + p_2(t)y(t) = q(t)$ and ICs $y(t_0) = y_0$, $y'(t_0) = y'_0$: If $f(t, y)$ and $\frac{\delta f}{\delta y} (t, y)$ are continuous in a rectangle around $(t_0, y_0)$, then a solution to IVP exists and is unique in some interval $(t_0-h, t_0+h)$.
    \end{itemize}
\end{enumerate}

%------------------------- Euler's Identity -----------------------
\textbf{Euler's Identity}
$e^{i\theta} = cos\theta + i sin\theta$

%------------------------- Euler's Method -----------------------
\textbf{Euler's Method}
Given $y' = f(t,y)$ and $y(t_0) = y_0$: $y(t+\Delta t) \approx y(t) + \Delta t f(t,y)$:
    \begin{itemize}
        \item At $t_0$: $v_0 = y(t_0)$.
        \item At $t_1 = t_0 + \Delta t$: $v_1 = v_0 + \Delta t f(t_0, v_0)$.
        \item At $t_n = t_0 + n\Delta t$: $v_n = v_{n-1} + \Delta t f(t_{n-1}, v_{n-1})$.
    \end{itemize}

%------------------------- Method of Separation -----------------------

\textbf{Method of Separation of Variables}
\begin{enumerate}
    \item Given $\frac{dy}{dt} = G(y)H(t)$, rewrite as $\frac{1}{G(y)}dy = H(t)dt$.
    \item Take the integral of both sides to get $\int \frac{1}{G(y)}dy = \int H(t)dt$.
\end{enumerate}

%------------------------- Method of Integrating Factors -----------------------

\textbf{Method of Integrating Factors}
\begin{enumerate}
    \item Given ODE, put in normal form $y'(t) + p(t)y(t) = q(t)$.
    \item Multiply ODE by integrating factor $\mu (t) = e^{\int p(t)dt}$.
    \item ODE becomes $\frac{d}{dt} [\mu (t) y(t)] = q(t)$, so take the integral of both sides to get $\mu (t) y(t) = \int q(t)$. Divide both sides by $\mu (t)$.
\end{enumerate}

%------------------------- Solving 2nd Order Constant Coefficient Homogeneous DEs -----------------------
\textbf{Solving 2nd Order Constant Coefficient Homogeneous DEs}
Given $a y''(t) + b y'(t) + c y(t) = 0$, solve characteristic polynomial $a\lambda ^2+b\lambda +c = 0$ for roots $\lambda _1$ and $\lambda _2$. If no roots are repeated, then by superposition, $y(t) = C_1e^{\lambda_1 t}+C_2e^{\lambda_2 t}$. If roots are repeated, then $y(t) = C_1e^{\lambda_1 t}+C_2te^{\lambda_1 t}$.

%------------------------- Variation of Parameters -----------------------
\textbf{Variation of Parameters}
\begin{enumerate}
    \item Find the general solution to the homogeneous version of the ODE, where $f(t) = 0$.
    \item Suppose the homogeneous solution has form $y(t) = u_1(t)y_1(t)+u_2(t)y_2(t)$. Then, compute the Wronskian with the form $W = \det {\begin{vmatrix}y_1&y_2\\y'_1&y'_2 \end{vmatrix}}$. If $W(y_1, y_2) \neq 0$, then $u_1 = -\int \frac{y_2f}{W} dt$ and $u_2 = \int \frac{y_1f}{W} dt$.
\end{enumerate}

%------------------------- Reduction of Order -----------------------
\textbf{Reduction of Order}
\begin{itemize}
    \item From Abel's Theorem, we know $W[y_1, y_2] = y_1y'_2-y_2y'_1 = Ce^{-\int p(t) dt}$. \textbf{or}
    \item Ansatz $y_2 = y(t)y_1$; plug into ODE and solve for u(t). Set $w = u'$ and solve ODE before plugging $u'$ back into $w$.
\end{itemize}

%------------------------- Abel's Theorem -----------------------
\textbf{Abel's Theorem}
If p, q are continuous functions an open interval \textit{I}, and $y_1, y_2$ are solutions to $y'' + py' + qy = 0$, then for some constant $C$, $W[y_1, y_2] = Ce^{-\int p(t) dt}$. For $y_1, y_2$ solutions on $I$ for $y''+py'+qy=0$, see that either $W \equiv 0$ (always 0) or $W \neq 0$ for all $t \in I$ (never 0).

%------------------------- Linear Independence of Functions -----------------------
\textbf{Linear Independence of Functions}
\begin{itemize}
    \item We say {$f_1(t),...,f_k(t)$} is linearly independent set on an open interval $(a, b)$ if the only solution to $c_1f_1(t) + ... + c_kf_k(t) = 0$ for all $t \in (a, b)$ is with \textit{all} $x_i = 0$.
    \item Two functions $\{f, g\}$ is a linearly independent set on an open interval $(a, b)$ if $W[f, g] \neq 0$ for \textit{some} $t \in (a, b)$.
    \item A pair of functions $\{y_1, y_2\}$ solving $y'' + py' + qy = 0$ form a fundamental solution set on an open interval $I$ if $W[y_1, y_2] \neq 0$ for \textit{some} $t \in I$.
\end{itemize}

%------------------------- Method of Undetermined Coefficients -----------------------
\textbf{Method of Undetermined Coefficients}
\begin{enumerate}
    \item Find the general solution $y_h$ to the homogeneous version of the ODE, where $f(t) = 0$.
    \item Guess a solution $y_p$ to the forced ODE; plug into ODE to verify.
If any term in the first guess is also a solution to the corresponding homogeneous equation, multiply the whole guess by t. If any term in this second guess is still a solution to the homogeneous equation, multiply by t again (i.e. multiply the first guess by $t^2$).
    \item The general solution to the forced ODE is $y(t) = y_h + y_p$.
\end{enumerate}

\begin{center}
    \begin{tabular}{ || c | c || } 
         \hline \hline
         $f(t)$ & $y_p(t)$ First Guess \\ 
         \hline \hline
         $ke^{rt}$ & $Ae^{rt}$ \\ 
         \hline
         $kcos(\omega t)$ or $ksin(\omega t) $ & $Asin(\omega t)+Bcos(\omega t)$ \\ 
         \hline
         $P_n(t)$ & $A_nt^n + A_{n−1}t^{n−1} + ... + A_1t + A_0$ \\
         \hline
         $P_n(t)e^{rt}$ & $(A_nt^n + A_{n−1}t^{n−1} + ... + A_1t + A_0)e^{rt}$ \\
         \hline
         $P_n(t)e^{rt}sin(\omega t)$ or $P_n(t)e^{rt}cos(\omega t)$ & $(A_nt^n + A_{n−1}t^{n−1} + ... + A_1t + A_0)e^{rt}sin(\omega t)$ \\
         & $+ (B_nt^n + B_{n−1}t^{n−1} + ... + B_1t + B_0)e^{rt}cos(\omega t)$ \\
         \hline \hline
    \end{tabular}
\end{center}
%------------------------- Springs -----------------------
\textbf{Springs}
\begin{itemize}
    \item For a mass spring, we have equation $mx"+cx'+kx=kX(t)$ where $X(t)$ refers to the forcing term. Can be $X(t)=X_0 cos(\omega _f t)$ where $\omega _f$ is the forcing frequency.
    \item For an undamped, unforced mass spring of equation $mx"+kx=0$ or $x"+\frac{k}{m}x=0$ where $\frac{k}{m}=\omega^2$ and $\omega_n$ is the natural frequency, the solution is $x(t)=Asin(\omega t)+Bcos(\omega t)=\sqrt{A^2+B^2}sin(\omega t+tan^{-1}(\frac{A}{B}))$.
    \item For an undamped, forced mass spring of equation $x"+\omega^2x=\omega^2X_0cos(\omega_ft)$, the general solution is $x(t)=Asin(\omega t)+Bcos(\omega t)+\frac{\omega^2X_0}{\omega^2-\omega_f^2}cos(\omega_f t)$. For IC's x(0)=0 and x'(0)=0, then the solution is $x(t)=\frac{\omega^2X_0}{\omega^2-\omega_f^2}[cos(\omega_f t)-cos(\omega t)]$.
    \item For a damped free oscillator of equation $m\Ddot{x}+c\Dot{x}+kx=0$, its natural frequency is $\omega=\sqrt{\frac{k}{m}}$ when undamped. Solving its characteristic equation yields roots $\lambda_{1, 2}=\frac{-c\pm\sqrt{c^2-4mk}}{2m}$, where $c^2-4mk$ is the discriminant $\Delta$.
        \begin{center}
            \begin{tabular}{ | c | c | c | c | } 
                 \hline
                 $\zeta > 1$ & $\Delta > 0$ & distinct roots $\lambda_1,\lambda_2$ & "overdamped" $x_h(t)=C_1e^{\lambda_1 t}+C_2e^{\lambda_2 t}$ \\
                 \hline
                 $\zeta = 1$ & $\Delta = 0$ & repeated roots $\lambda$ & "critically damped" $x_h(t)=C_1e^{\lambda t}+C_2te^{\lambda t}$ \\
                 \hline
                 $0 < \zeta < 1$ & $\Delta < 0$ & complex roots $\lambda_1,\lambda_2 = \sigma \pm i \omega_d$ & "underdamped"  $x_h(t)=e^{\sigma t}[Asin(\omega_d t)+Bcos(\omega_d t)]$ \\
                 \hline
                 $\zeta = 0$ & $\Delta < 0$ & purely imaginary $\lambda_1,\lambda_2 = \pm i \omega_d$ & "undamped" $x_h(t)=Asin(\omega_d t)+Bcos(\omega_d t)$ \\
                 \hline
            \end{tabular}
            \newline
            $\omega_d$ is the damping frequency, where if $\lambda_{1, 2} = -\zeta\omega\pm i\sqrt{1-\zeta}\omega$, then $\omega_d=\sqrt{1-\zeta}\omega$. We can rewrite $\Ddot{x}+\frac{c}{m}\Dot{x}+\frac{k}{m}x=0$ as $\Ddot{x}+2\zeta\omega\Dot{x}+\omega^2x=0$ if we let $\zeta=\frac{c}{2\sqrt{mk}}$. 
        \end{center}
    \item For a damped forced oscillator, see method of undetermined coefficients.
\end{itemize}

%------------------------- Systems of ODEs -----------------------
\textbf{Systems of ODEs}
To rewrite higher order ODEs as a system of first order ODEs:
\begin{enumerate}
    \item Rewrite the ODE so that the highest order term is isolated.
    \item Set a function equal to the next highest order term.
    \item Rewrite the ODE to have all the terms written in terms of the new function.
    \item Rinse and repeat.
    \item If and only if the original ODE is linear, rewrite the system as a matrix.
\end{enumerate}

%------------------------- Systems of ODEs Example -----------------------
\textbf{Systems of ODEs Example 1}
\newline
$y^2y''' + 7y = 0$ \\
Rewrite as $y''' =-7y^{-1}$. \\
Let $v=y'$, so $v"=-7y^{-1}$. \\
Let $u=v'$, so $u'=-7y^{-1}$. \\
Then, the system of equations is: $\{y'=v, v'=u, u'=-7y^{-1}\}$.

%------------------------- Systems of ODEs Example -----------------------
\textbf{Systems of ODEs Example 2}
\newline
$4y"+3y'+2y=0$ \\
Rewrite as $y"=-\frac{3}{4}y'-\frac{1}{2}y$. \\
Let $u=y'$, so $u'=-\frac{3}{4}u-\frac{1}{2}y$. \\
Then, the system of equations is: $\{y'=u, u'=-\frac{3}{4}u-\frac{1}{2}y\}$. \\
In matrix form: \\
$\begin{bmatrix}
y' \\
u' \\
\end{bmatrix}$
=
$\begin{bmatrix}
0 & 1 \\
-\frac{1}{2} & -\frac{3}{4} \\
\end{bmatrix}$
$\begin{bmatrix}
y \\
u \\
\end{bmatrix}$.

\end{document}